\documentclass[letterpaper, 12pt]{article}
\usepackage[utf8]{inputenc}
\usepackage{amsmath,amssymb,amsfonts}
\usepackage[spanish,es-tabla]{babel}
\spanishdecimal{.} % Usar . como separador de decimal
\usepackage{algorithmic}
\usepackage{graphicx}
\usepackage{subcaption}
\usepackage{textcomp}
\usepackage{xcolor}
\usepackage{enumerate}
\usepackage{tikz} % Dibujar resortes
\usetikzlibrary{decorations.pathmorphing,patterns}%también para los resortes xd
\usepackage[left=1in, right=1in, top=1in, bottom=1in]{geometry}

\usepackage{float} % Ubicar figuras HERE
\usepackage{soul}  % Resaltador

\title{\protect\vspace{-0.5cm}\protect\includegraphics[width =6cm]{Logo UIS.png} \\
Primer Parcial - Dinámica de Procesos}
\author{Juan Sebastián Guerrero Peña - 2190435 - B1\and
        Luis Fernando Romero Rojas - 2191663 - B1\and
        Rafael Santiago Suarez Gil - 2190433 - B2\and
        Sebastián García Angarita - 2191472 - B2\\ \\
        \textit{Escuela de Ingeniería Eléctrica, Electrónica y Telecomunicaciones} \\
        \textit{Universidad Industrial de Santander}\\
        Bucaramanga, Colombia \\}
\date{}

\begin{document}
\maketitle
\begin{center}\rule{0.9\linewidth}{0.5pt}\end{center}
\parskip = 12pt

Todas las unidades de las variables usadas en algunos de los ejercicios corresponden a las establecidas en el sistema internacional de unidades (SI). En la siguiente tabla se resumen

\begin{table}[H]
\centering
\begin{tabular}{l|c|c}
    \hline
    Nombre \hspace{3cm} & Símbolo & Unidades (SI) \\
    \hline
    Masa & $M$ & $\mathrm{kg}$ \\
    Constante de Resorte & $K$ & $\mathrm{N/m}$ \\
    Velocidad Angular & $\omega$ & $\mathrm{rad/s}$ \\
    Momento de Inercia & $J$ & $\mathrm{kg\cdot m^2}$ \\
    
\end{tabular}
\caption{Unidades de las variables usadas en los ejercicios.}
\label{tab:my_label}
\end{table}

\begin{enumerate}
\item Desarrollen numéricamente los siguientes modelos matemáticos, grafique los resultados y analícelos:
\begin{enumerate}
\item Utilicen los siguientes valores:
$L = 1.21$; $g = 10$; $a_1 = 2.5$; $a_2 = 5.0$. Asuma condiciones iniciales diferentes de cero para $t(0)$. El modelo es dimensionalmente consistente:
\begin{equation}
\begin{split}
    L^2\ddot{\varphi} - \ddot{\gamma}\sin \varphi + \ddot{\mu}\cos \varphi + \cos \varphi &= 0 \\
    \frac{a_1+a_2}{a_2L}\ddot{\mu} - \ddot{\varphi}\cos \varphi - \dot{\varphi}^2\sin \varphi &= - \frac{a_1+a_2}{a_2L}g \\
    \frac{a_1+a_2}{a_2L}\ddot{\gamma} - \ddot{\varphi}\sin \varphi &= \dot{\varphi}^2\cos \varphi
\end{split}
\end{equation}

% SOLUCIÓN
\textbf{Solución}

Para solucionar el sistema se usó MATLAB y Simulink, se establecieron las siguientes condiciones iniciales:
$\varphi(0) = 2$; $\dot{\varphi}(0) = 1$; $\mu(0) = 1$; $\dot{\mu}(0) = 2$; $\gamma(0) = 3$; $\dot{\gamma}(0) = 1$.

\begin{figure}[H]
    \centering
    \hspace{1cm}
    \begin{subfigure}[h]{0.45\linewidth}
        \includegraphics[width=\linewidth]{plots 1a/grafico de varphi.png}
        \caption{\centering Gráfica $\varphi(t)$}
    \end{subfigure}
    \begin{subfigure}[h]{0.45\linewidth}
        \includegraphics[width=\linewidth]{plots 1a/grafico de mu.png}
        \caption{\centering Gráfica $\mu(t)$}
    \end{subfigure}
    \begin{subfigure}[h]{0.45\linewidth}
        \includegraphics[width=\linewidth]{plots 1a/grafico de gamma.png}
        \caption{\centering Gráfica $\gamma(t)$}
    \end{subfigure}
    \caption{Gráficas solución del sistema}
    \label{fig:graficas solucion del sistema 1a}
\end{figure}

Vemos que la variable $\varphi(t)$ tiene un comportamiento periódico de una forma sinusoidal, mientras que la variable $\mu(t)$ tiende a una parábola. A diferencia, la variable $\gamma(t)$ parece estar compuesta por una componente periódica y otra lineal con el tiempo.
Estas descripciones dadas, son aproximaciones dadas a lo visto en las gráficas de la figura \ref{fig:graficas solucion del sistema 1a}, y el comportamiento real de las variables puede variar para rangos de tiempo diferente o condiciones iniciales distintas.






\item $g = 10$; $m_1 = 10$; $m_2 = 30$; $k = 30$; $r_1 = 2.0$. El modelo es dimensionalmente consistente.
\begin{equation}
\begin{split}
    \left(m_1{r_1}^2 + m_2{r_2}^2\right)\ddot{\theta} + 2m_2r_2\dot{r_2}\dot{\theta} + \left(m_1r_1 + m_2r_2\right)g\sin \theta &= 0 \\
    m_2\ddot{r_2} - m_2r_2{\dot{\theta}}^2 - m_2g\cos \theta + k\left(r_2 - 1.1\right) &= 0
\end{split}
\end{equation}









\item Para el siguiente modelo asuma que $J_1 < J_2 < J_3$. Proponga valores para estos momentos de inercia cumpliendo la desigualdad y analice. Realice al menos cinco simulaciones con diferentes $J$'s manteniendo las mismas condiciones iniciales (que deben ser diferentes a cero, para el tiempo $t_0 = 0$). $\omega_i \equiv \omega_i(t)$,
\begin{equation}
\begin{split}
    J_1\dot{\omega_1} + \left( J_2 - J_3\right)\omega_2\omega_3 &= 0 \\
    J_2\dot{\omega_2} - \left( J_1 - J_3\right)\omega_1\omega_3 &= 0 \\
    J_3\dot{\omega_3} + \left( J_1 - J_2\right)\omega_2\omega_1 &= 0
\end{split}
\end{equation}

% SOLUCIÓN
\textbf{Solución}

Estas ecuaciones describen la rotación de un objeto que tiene tres momentos de inercia distintos en sus tres ejes principales.
Cuando el objeto rota únicamente sobre uno de los ejes, se dice que el objeto está en equilibrio porque la velocidad angular no cambia y permanece igual que la velocidad inicial como se puede ver en la figura \ref{fig:sistema en punto de equilibrio}.

\begin{figure}[H]
    \centering
    \hspace{1cm}
    \begin{subfigure}[h]{0.28\linewidth}
        \includegraphics[width=\linewidth]{plots 1c/w0=[1,0,0].png}
        \caption{\centering Rotación en el primer eje}
    \end{subfigure}
    \begin{subfigure}[h]{0.28\linewidth}
        \includegraphics[width=\linewidth]{plots 1c/w0=[0,1,0].png}
        \caption{\centering Rotación en el segundo eje}
    \end{subfigure}
    \begin{subfigure}[h]{0.28\linewidth}
        \includegraphics[width=\linewidth]{plots 1c/w0=[0,0,1].png}
        \caption{\centering Rotación en el tercer eje}
    \end{subfigure}
    \caption{Velocidades angulares cuando el sistema está en un punto de equilibrio}
    \label{fig:sistema en punto de equilibrio}
\end{figure}

Estos resultados son intuitivos, sin presencia de fuerzas externas el cuerpo permanece rotando sobre el mismo eje a la misma velocidad.

Ahora un caso más interesante, es cuando el cuerpo rota sobre el eje intermedio, esta rotación es inestable y  se afecta con incluso pequeñas perturbaciones. Este es el conocido como teorema del eje intermedio o efecto de Dzhanibekov.

Para fines del ejercicio, se establecen las siguientes velocidades iniciales de cada eje:
$\omega_1(0) = 0.001$; $\omega_2(0) = 1$; $\omega_3(0) = 0.001$.
Y se analiza el comportamiento de las rotaciones al cambiar el momento de inercia de los ejes.
Se establecieron las 5 siguientes distintas configuraciones:

\begin{table}[H]
\centering
\begin{tabular}{|c|c|c|c|}
    \hline
    Configuración & $J_1$ & $J_2$ & $J_3$ \\
    \hline
    1   & $2$   & $4$   & $7$   \\
    2   & $1$   & $4$   & $7$   \\
    3   & $2$   & $5$   & $7$   \\
    4   & $2$   & $3$   & $7$   \\
    5   & $2$   & $4$   & $9$   \\
    \hline
\end{tabular}
\caption{Distintas configuraciones de momentos de inercia}
\label{tab:configuraciones de momentos de inercia}
\end{table}

En la Figura \ref{fig:configuraciones de momentos de inercia} se grafican las velocidades angulares de cada eje para cada una de las configuraciones de momentos de inercia.

\begin{figure}[H]
    \centering
    \begin{subfigure}[H]{0.45\linewidth}
        \includegraphics[width=\linewidth]{plots 1c/w_i vs t - v1.png}
        \caption{\centering Configuración 1}
        \label{fig:config 1}
    \end{subfigure}
    \hspace{0.5cm}
    \begin{subfigure}[H]{0.45\linewidth}
        \includegraphics[width=\linewidth]{plots 1c/w_i vs t - v2.png}
        \caption{\centering Configuración 2}
        \label{fig:config 2}
    \end{subfigure}
    \newpage
    \begin{subfigure}[H]{0.45\linewidth}
        \includegraphics[width=\linewidth]{plots 1c/w_i vs t - v3.png}
        \caption{\centering Configuración 3}
        \label{fig:config 3}
    \end{subfigure}
    \hspace{0.5cm}
    \begin{subfigure}[H]{0.45\linewidth}
        \includegraphics[width=\linewidth]{plots 1c/w_i vs t - v4.png}
        \caption{\centering Configuración 4}
        \label{fig:config 4}
    \end{subfigure}
    \\
    \begin{subfigure}[H]{0.45\linewidth}
        \includegraphics[width=\linewidth]{plots 1c/w_i vs t - v5.png}
        \caption{\centering Configuración 5}
        \label{fig:config 5}
    \end{subfigure}
    \caption{Velocidades angulares cuando el sistema está en un punto de equilibrio}
    \label{fig:configuraciones de momentos de inercia}
\end{figure}

Tomando la configuración 1 como la configuración de referencia. Si se hace una comparación con la configuración 2, en este caso el momento de inercia $J_1$ disminuye. En principio se sabe que para un menor momento de inercia es más fácil rotar y obtener una mayor velocidad angular, y como se puede comparar en las gráficas \ref{fig:config 1} y \ref{fig:config 2}, la velocidad angular $\omega_1$ alcanza un valor máximo mayor respecto a la configuración base.
Como el efecto de Dzhanibekov ocurre en ese primer eje con menor momento de inercia. Si ahora el momento de inercia es menor, es más fácil hacerlo rotar en ese eje para la misma pequeña perturbación, por lo que es válido afirmar que va a rotar con mayor frecuencia como se ve en las gráficas.
\\ \\
En resumen, de todas las configuraciones se puede concluir:
\begin{itemize}
    \item Las rotaciones en puntos de equilibrio son estables, el objeto permanece rotando con velocidad constante sobre el mismo eje.
    \item La rotación sobre el eje intermedio es inestable, una pequeña perturbación puede hacer rota el objeto 
    \item Cuando la diferencia entre $J_1$ y $J_2$ mayor, es cuando el sistema es más inestable. Como el objeto principalmente rota sobre el eje intermedio, si $J_1$ es mucho menor, es más fácil que empiece a rotar en ese eje.
    \item Cuando la diferencia de momentos de inercia del primer eje $J_1$ y el tercer eje $J_3$ es mayor, el efecto ocurre más frecuente, pero como el objeto rota principalmente sobre el eje intermedio, el impacto que tiene esta mayor diferencia de momentos de inercia es menor comparado con el caso anterior.
\end{itemize}


\end{enumerate}
\item Representen es ESPACIO DE ESTADOS (en caso de que se pueda) los siguientes modelos:
\begin{enumerate}
\item Se medirán todas las salidas con sendos sensores junto con las fuentes.
\begin{equation}
\begin{split}
    D\ddot{\varphi} - \ddot{\gamma}\sin \varphi + \ddot{\mu}\cos \varphi + \cos \varphi &= 0 \\
    A\ddot{\mu} - \ddot{\varphi}\cos \varphi - {\dot{\varphi}}^2\sin \varphi &= C \\
    B\ddot{\gamma} - \ddot{\varphi}\sin \varphi &= {\dot{\varphi}}^2\cos \varphi
\end{split}
\end{equation}

% SOLUCIÓN
\textbf{Solución}

En las ecuaciones del sistema se nota que en una misma ecuación aparecen varias derivadas del mayor orden. Para hacer una representación, es necesario transformar el sistema a su forma normal, por lo que es necesario manipular las ecuaciones.

Se procede a despejar $\ddot{\mu}$ y $\ddot{\gamma}$ de la segunda y tercera ecuación respectivamente

\begin{equation*}
    \ddot{\mu} = \frac{1}{A}\left(C + \ddot{\varphi}\cos \varphi + {\dot{\varphi}}^2\sin\varphi\right)
\end{equation*}
\begin{equation*}
    \ddot{\gamma} = \frac{1}{B}\left( \ddot{\varphi}\sin\varphi + {\dot{\varphi}}^2\cos\varphi \right)
\end{equation*}

También es necesario estableces algunas variables de fase:
\begin{align*}
    x_1 &= \varphi & x_3 &= \mu & x_5 &= \gamma \\
    x_2 &= \dot{x}_1 = \dot{\varphi} & x_4 &= \dot{x}_3 = \dot{\mu} & x_6 &= \dot{x}_5 = \dot{\gamma} \\
\end{align*}

Al remplazar en la primera ecuación del sistema se puede obtener $\ddot{\varphi}$ siendo esta la derivada de mayor orden:

\begin{equation}
    \dot{x}_2 = \frac{AB}{ABD - A\sin^2 x_1 + B\cos^2 x_1} \left[ \frac{A-B}{AB}{x_2}^2\sin x_1\cos x_1 -\frac{A+C}{A}\cos x_1\right]
\end{equation}

Remplazando ahora $\dot{x}_2$ en $\dot{x}_4$ y $\dot{x}_6$
\begin{multline}
    \dot{x}_4 = \frac{ABD - A\sin^2 x_1 + B\cos^2 x_1 + (A^2-AB)\cos^2 x_1}{A^2BD-A^2\sin^2 x_1 + AB\cos^2 x_1} {x_2}^2\sin x_1 + \\
    \frac{ABCD - AC\sin^2 x_1 + BC\cos^2 x_1 - AB(A+C)\cos^2 x_1}{A^2BD - A^2\sin^2 x_1 + AB\cos^2 x_1}
\end{multline}
\begin{multline}
    \dot{x}_6 = \frac{1}{B}{x_2}^2\cos x_1 + \frac{A\sin x_1}{ABD-A\sin^2 x_1 + B\cos^2 x_1} \left[\frac{A-B}{AB}{x_2}^2\sin x_1\cos x_1 \right.\\ \left.-\frac{A+C}{A}\cos x_1 \right]
\end{multline}

Dado que el sistema no es lineal, no es posible avanzar en la representación de espacio de estados, dado que no tiene representación matricial. Por tanto la representación final del sistema es:

\begin{equation*}
    \dot{x}_1 = x_2
\end{equation*}
\begin{equation*}
    \dot{x}_2 = \frac{AB}{ABD - A\sin^2 x_1 + B\cos^2 x_1} \left[ \frac{A-B}{AB}{x_2}^2\sin x_1\cos x_1 -\frac{A+C}{A}\cos x_1\right]
\end{equation*}
\begin{equation*}
    \dot{x}_3 = x_4
\end{equation*}
\begin{multline*}
    \dot{x}_4 = \frac{ABD - A\sin^2 x_1 + B\cos^2 x_1 + (A^2-AB)\cos^2 x_1}{A^2BD-A^2\sin^2 x_1 + AB\cos^2 x_1} {x_2}^2\sin x_1 + \\
    \frac{ABCD - AC\sin^2 x_1 + BC\cos^2 x_1 - AB(A+C)\cos^2 x_1}{A^2BD - A^2\sin^2 x_1 + AB\cos^2 x_1}
\end{multline*}
\begin{equation*}
    \dot{x}_5 = x_6
\end{equation*}
\begin{multline*}
    \dot{x}_6 = \frac{1}{B}{x_2}^2\cos x_1 + \frac{A\sin x_1}{ABD-A\sin^2 x_1 + B\cos^2 x_1} \left[\frac{A-B}{AB}{x_2}^2\sin x_1\cos x_1 \right.\\ \left.-\frac{A+C}{A}\cos x_1 \right]
\end{multline*}

\item Aquí se medirán simultáneamente todas las salidas con sendos sensores, e incluso las mismas entradas (pero con sensores diferentes y únicos (¡dedicación exclusiva para cada entrada!)).

\begin{equation}
\begin{split}
    55\dddot{x}_1-\ddot{x}_2-7\dot{x}_3-5x_2-7\sin(0.1t)=0 \\
    {\dddot{x}}_2-\ {\ddot{x}}_1\ +18{\dot{x}}_2\ +\ 4x_3\ +\ 7\cosh(0.5t)=0 \\
    \pi\dddot{x}_3\ -\ 6\ddot{x}_3\ -\ 7\dot{x}_1\ -\ 8x_2\ -\ 3\tan(0.7t) =0 
\end{split}
\end{equation}

\textbf{Solución} \\
Identificamos 3 variables de estado las cuales estan todas como tercera derivadas por lo tanto necesitamos usar cambios de variable para obtener las 6 variables de fase que los complementan:
\begin{equation}
\begin{split}
    y_1 &= \dot{x}_1 \\
    y_2 &= \dot{x}_2 \\
    y_3 &= \dot{x}_3 \\
    z_1 &= \dot{y}_1 \\
    z_2 &= \dot{y}_2 \\
    z_3 &= \dot{y}_3 \label{2.b.CambioV}
\end{split} 
\end{equation}
Podemos entonces reescribir las ecuaciones como:
\begin{align*}
    55\dot{z}_1-z_2-7y_3-5x_2 &= 7\sin(0.1t) \\
    {\dot{z}}_2-\ z_1\ +18y_2\ +\ 4x_3\ &=\ -7 \cosh(0.5t) \\
    \pi\dot{z}_3\ -\ 6\ z_3\ -\ 7\ y_1\ -\ 8x_2\ &=\ 3\tan(0.7t) \   
\end{align*}
Si despejamos de cada una de las ecuaciones las derivadas de las variables de estado quedaría:
 \begin{equation}
 \begin{split}
    \dot{z}_1&=\frac{1}{55}z_2+\frac{7}{55}y_3+\frac{5}{55}x_2+\frac{7}{55}\sin(0.1t) \\
    {\dot{z}}_2&=\ z_1\ -18y_2\ -\ 4x_3\ \ -7\cosh(0.5t)  \\
    \dot{z}_3\ &=\ \frac{6}{\pi}\ z_3\ +\ \frac{7}{\pi}\ y_1\ +\ \frac{8}{\pi}x_2\ +\ \frac{8}{\pi}\tan(0.7t)
    \label{1.b.mainV}
\end{split}
\end{equation}

Si se usan las ecuaciones (\ref{2.b.CambioV}) y (\ref{1.b.mainV}) podemos formar la matriz: \
\begin{multline*}
\begin{bmatrix}
\dot{z}_1\\ 
\dot{z}_2\\ 
\dot{z}_3\\ 
\dot{y}_1\\ 
\dot{y}_2\\ 
\dot{y}_3\\ 
\dot{x}_1\\ 
\dot{x}_2\\ 
\dot{x}_3
\end{bmatrix}
=
\begin{bmatrix}
0 &1/55  &0  &0  &0  &7/55  &0  &5/55  &0 \\ 
1 &0  &0  &0  &-18  &0  &0  &0  &-4 \\ 
0 &0  &6/\pi &7/\pi   &0  &0  &0  &8/\pi   &0 \\ 
1 &0  &0  &0  &0  &0  &0  &0  &0 \\ 
0 &1  &0  &0  &0  &0  &0  &0  &0 \\ 
0 &0  &1  &0  &0  &0  &0  &0  &0 \\ 
0 &0  &0  &1  &0  &0  &0  &0  &0 \\ 
0 &0  &0  &0  &1  &0  &0  &0  &0 \\ 
0 &0  &0  &0  &0  &1  &0  &0  &0 
\end{bmatrix}
\begin{bmatrix}
z_1\\ 
z_2\\ 
z_3\\ 
y_1\\ 
y_2\\ 
y_3\\ 
x_1\\ 
x_2\\ 
x_3
\end{bmatrix}
\cdots \\
\cdots +
\begin{bmatrix}
7/55 &0  &0   \\ 
0 &-7  &0   \\ 
0 &0  &3/\pi \\
0  &0  &0 \\
0  &0  &0 \\
0  &0  &0 \\
0  &0  &0 \\
0  &0  &0 \\
0  &0  &0 \\
\end{bmatrix}
\begin{bmatrix}
\sin(0.1t)\\ 
\cosh(0.5t)\\ 
\tan(0.7t)
\end{bmatrix}
\end{multline*}
La cual tiene la forma de la \textbf{ecuación de estado}: $\dot{x}=Ax+Bu$. \\
Como se requiere medir todas las salidas y todas las entradas por independiente nuestra ecuación de salida tiene que ser un vector de $6\cdot 1$ donde 3 espacios son ocupados por las variables de estado $x_1, x_2$ y $x_3$ y 3 por las entradas, tendríamos entonces:


\begin{multline*}
\begin{bmatrix}
\ y_1\\ 
\ y_2\\ 
\ y_3\\ 
\ y_4\\ 
\ y_5\\ 
\ y_6
\end{bmatrix}
=
\begin{bmatrix}
0 &0  &0  &0  &0  &0  &1  &0  &0 \\ 
0 &0  &0  &0  &0  &0  &0  &1  &0 \\ 
0 &0  &0  &0  &0  &0  &0  &0  &1 \\ 
0 &0  &0  &0  &0  &0  &0  &0  &0 \\ 
0 &0  &0  &0  &0  &0  &0  &0  &0 \\ 
0 &0  &0  &0  &0  &0  &0  &0  &0 
\end{bmatrix}
\begin{bmatrix}
z_1\\ 
z_2\\ 
z_3\\ 
y_1\\ 
y_2\\ 
y_3\\ 
x_1\\ 
x_2\\ 
x_3
\end{bmatrix}
\cdots \\
\cdots +
\begin{bmatrix}
0  &0  &0 \\
0  &0  &0 \\
0  &0  &0 \\
1  &0  &0 \\
0  &1  &0 \\
0  &0  &1 
\end{bmatrix}
\begin{bmatrix}
\sin(0.1t)\\ 
\cosh(0.5t)\\ 
\tan(0.7t)
\end{bmatrix}
\end{multline*}



\item Se medirán simultáneamente todas las salidas con sendos sensores (incluyendo las entradas).
\begin{equation}
\begin{split}
    12{\dddot{x}}_1\ -\ {\ddot{x}}_2\ +\ t{\dot{x}}_5\ -\ 5{\dot{x}}_4\ -\ \sin(0.1t) &=0 \\
    \dddot{x}_2\ -\ t\ddot{x}_6\ +\ 18\dot{x}_2\ + 4x_3 + x_1 + t \cosh(0.5) &= 0 \\
    6{\ddot{x}}_5-\ t^{-1}{\dot{x}}_1+\ 8{\ddot{x}}_2-\ 3\tan(t + 1)x_6 &= 0\\
    5{\dddot{x}}_4\ -\ {\ddot{x}}_2\ -\ 7{\ddot{x}}_3\ -\cos(t)x_1\ +\ 0.1 &= 0 \\ 
    12{\dddot{\ x}}_5-\ {\ddot{x}}_1+\ t^{3}{\dot{x}}_2+\ 4x_4+\ x_3\cosh(t) &=0 \\
    -t{\dddot{x}}_6\ -\ e^{-5t}{\ddot{x}}_3-\ 3{\dot{x}}_1-\ 8x_5+\ t\tan(t + 2) &= 0
    \label{2.c:main}
\end{split}
\end{equation}

\textbf{Solución} \\ 
Inspeccionando las ecuaciones vemos que todas son de orden 3 a exepcion de la tercera ecuación, esta tienen un orden de dos tanto para $x_5$ como para $x_2$ como orden máximo, para poder repesentar en variable de estados requerimos despejar la de mayor orden para que quede en términos de las demás, elegiremos $x_5$ para ser despejada. Al despeja todas las ecuaciones de (\ref{2.c:main}) de la forma mencionada, nos queda:

\begin{equation}
\begin{split}
    {\dddot{x}}_1\ &=\ \frac{1}{12}{\ddot{x}}_2\ -\ \frac{t}{12}{\dot{x}}_5\ +\frac{5}{12}{\dot{x}}_4+ \frac{1}{12} \sin(0.1t)  \\
    \dddot{x}_2\ &= \ t\ddot{x}_6\ -\ 18\dot{x}_2\ - 4x_3 - x_1 - t \cosh(0.5) \\
    {\ddot{x}}_5\ &=\ \frac{t^{-1}}{6}{\dot{x}}_1-\ \frac{8}{6}{\ddot{x}}_2+\ \frac{3}{6}\tan(t + 1)x_6 \\
    {\dddot{x}}_4\ &=\ \frac{1}{5}{\ddot{x}}_2\ +\ \frac{7}{5}{\ddot{x}}_3\ +\frac{1}{5}\cos(t)x_1\ -\frac{0.1}{5}  \\ 
    {\dddot{\ x}}_5\ &=\ \frac{1}{12}{\ddot{x}}_1-\ \frac{t^{3}}{12}{\dot{x}}_2-\ \frac{4}{12}x_4-\ \frac{1}{12}x_3\cosh(t)  \\
    {\dddot{x}}_6\ &= - \frac{e^{-5t}}{t} {\ddot{x}}_3-\ \frac{3}{t}{\dot{x}}_1-\ \frac{8}{t}x_5+\ \tan(t + 2) 
    \label{2.c:despejada}
\end{split}
\end{equation}

Identificando las 6 variables de estado del sistema. necesitamos hacer cambios de variable para tener las variables de fase:

\begin{equation}
\begin{split}
    y_1 &= \dot{x}_1 \\
    y_2 &= \dot{x}_2 \\
    y_3 &= \dot{x}_3 \\
    y_4 &= \dot{x}_4 \\
    y_5 &= \dot{x}_5 \\
    y_6 &= \dot{x}_6 \\
    z_1 &= \dot{y}_1 \\
    z_2 &= \dot{y}_2 \\
    \mathbf{z_3} &\mathbf{= \dot{y}_3} \\ %Esta negrilla cambia el tipo de letra
    z_4 &= \dot{y}_4 \\
    z_5 &= \dot{y}_5 \\
    z_6 &= \dot{y}_6  \label{2.c.CambioV}
\end{split} 
\end{equation}
Haciendo los respectivos cambios de variable de (\ref{2.c.CambioV}) en (\ref{2.c:despejada}) nos queda: 

\begin{equation}
\begin{split}
    {\dot{z}}_1\ &=\ \frac{1}{12}z_2\ -\ \frac{t}{12}y_5\ +\frac{5}{12}y_4+ \frac{1}{12} \sin(0.1t)  \\
    {\dot{z}}_2\ &= \ t\ z_6\ -\ 18y_2\ - 4x_3 - x_1 - t \cosh(0.5) \\
    {\dot{y}}_5\ &=\ \frac{t^{-1}}{6}y_1-\ \frac{8}{6}z_2+\ \frac{3}{6}\tan(t + 1)x_6 \\
    {\dot{z}}_4\ &=\ \frac{1}{5}z_2\ +\ \frac{7}{5}\mathbf{{\dot{y}}_3}\ +\frac{1}{5}\cos(t)x_1\ -\frac{0.1}{5}  \\ 
    {\dot{z}}_5\ &=\ \frac{1}{12}z_1-\ \frac{t^{3}}{12}y_2-\ \frac{4}{12}x_4-\ \frac{1}{12}x_3\cosh(t)  \\
    {\dot{z}}_6\ &= - \frac{e^{-5t}}{t} \mathbf{{\dot{y}}_3}-\ \frac{3}{t}{y}_1-\ \frac{8}{t}x_5+\ \tan(t + 2)
    \label{2.c:remplazada}
\end{split}
\end{equation}

Notece que al no existir $\dddot{x}_3$ no existe $\dot{z}_3$ esto implica no se pueda usar la ecuación resaltada en (\ref{2.c.CambioV}) ya que $x_3$ no se puede bajar mas de grado.Es un sistema de 17 ecuaciones y 18 incógnitas. Esto nos imposibilita a sacar la matriz de Espacio de estados. y por tanto no se puede derminar \textbf{represetancion de espacio de estados}



\end{enumerate}
\item Linealice los siguientes modelos (en caso de ser no-lineales) y resuélvanlos numéricamente. Definan las condiciones de solución del sistema. Grafiquen las soluciones del sistema no-lineal y el
correspondiente linealizado, y analicen. Incluya todos los pasos de la linealización. 
\begin{enumerate}
\item Linealizarlo alrededor del punto de operación: $\bar{x}_1=\bar{x}_2=5$.
\begin{equation}
\begin{split}
    \dddot{x}_1-\ddot{x}_1+18\ddot{x}_2+4x_1+x_2+\cosh(x_1) &= 0 \\ \dddot{x}_2-\ddot{x}_2+18\ddot{x}_1+4x_2+x_1+\sinh(x_2) &= 0
\end{split}
\label{mod1}
\end{equation}

% SOLUCIÓN
\textbf{Solución:}
El modelo es no lineal debido al coseno y al seno hiperbólico que afecta a las variables dependientes, para linealizar al rededor de $\bar{x}_1=\bar{x}_2=5$ hacemos $x_1=\delta x_1+5$ y $x_2=\delta x_2+5$ donde $\delta x_1$ y $\delta x_2$ son pequeñas excursiones al rededor de $\bar{x}_1=\bar{x}_2=5$

\begin{equation}
\notag
\begin{split}
    \frac{d^3(\delta x_1 + 5)}{dt^3}-\frac{d^2(\delta x_1 + 5)}{dt^2}+18\frac{d^2(\delta x_2 + 5)}{dt^2}+4(\delta x_1 + 5)+\delta x_2 + 5+\cosh(\delta x_1 + 5) &= 0 \\ \frac{d^3(\delta x_2 + 5)}{dt^3}-\frac{d^2(\delta x_2 + 5)}{dt^2}+18\frac{d^2(\delta x_1 + 5)}{dt^2}+4(\delta x_2 + 5)+\delta x_1 + 5+\senh(\delta x_2 + 5) &= 0
\end{split}
\end{equation}

Simplificando obtenemos:

\begin{equation}
\begin{split}
    \dddot{\delta x_1}-\ddot{\delta x_1}+18\ddot{\delta x_2}+4\delta x_1 +\delta x_2 + 25+\cosh(\delta x_1 + 5) &= 0 \\ \dddot{\delta x_2}-\ddot{\delta x_2}+18\ddot{\delta x_1}+4\delta x_2 +\delta x_1 + 25+\senh(\delta x_2 + 5) &= 0 
\end{split}
\label{nuevomod3a}
\end{equation}

Para linealizar las funciones seno y coseno hiperbólico hacemos uso de las series de Taylor: 

\begin{equation}
    \notag
    \cosh(\delta x_1 + 5)\simeq \frac{\cosh(5)}{0!}+\frac{\senh(5)}{1!}\left(\delta x_5\right)+\frac{\cosh(5)}{2!}\left(\delta x_1\right)^2+\frac{\senh(5)}{3!}\left(\delta x_1\right)^3+ \cdots
\end{equation}
\begin{equation}
    \notag
    \senh(\delta x_2 + 5)\simeq \frac{\senh(5)}{0!}+\frac{\cosh(5)}{1!}\left(\delta x_2\right)+\frac{\senh(5)}{2!}\left(\delta x_2\right)^2+\frac{\cosh(5)}{3!}\left(\delta x_2\right)^3+ \cdots
\end{equation}

Ya que buscamos linealizar el modelo tomaremos solamente los dos primeros términos de la sumatoria, es decir:
\begin{equation}
    \notag
    \cosh(\delta x_1 + 5)\simeq \frac{\cosh(5)}{0!}+\frac{\senh(5)}{1!}\left(\delta x_1\right)
\end{equation}
\begin{equation}
    \notag
    \senh(\delta x_2 + 5)\simeq \frac{\senh(5)}{0!}+\frac{\cosh(5)}{1!}\left(\delta x_2\right)
\end{equation}

Al resolver obtenemos que

\begin{equation}
    \cosh(\delta x_1 + 5)\simeq 74.203\delta x_1+74.21
    \label{cosh}
\end{equation}

\begin{equation}
    \senh(\delta x_2 + 5)\simeq 74.21\delta x_2+74.203
    \label{senh}
\end{equation}

Reemplazamos (\ref{cosh}) y (\ref{senh}) en (\ref{nuevomod3a1}):

\begin{equation}
\begin{split}
    \dddot{\delta x_1}-\ddot{\delta x_1}+18\ddot{\delta x_2}+4\delta x_1 +\delta x_2 +74.203\delta x_1+99.21 &= 0 \\ \dddot{\delta x_2}-\ddot{\delta x_2}+18\ddot{\delta x_1}+4\delta x_2 +\delta x_1 +74.21\delta x_2+99.203 &= 0 
\end{split}
\label{nuevomod3a1}
\end{equation}

Ya que $\delta x_1 = (x_1-\bar {x_1})$ y $\delta x_2 = (x_2-\bar{x_2})$ entonces el nuevo modelo linealizado sería 
\begin{equation}
\begin{split}
    \dddot{x_1}-\ddot{x_1}+18\ddot{x_2}+4x_1 + x_2 +78.203 x_1-296.805 &= 0 \\ \dddot{x_2}-\ddot{x_2}+18\ddot{x_1}+4x_2 +x_1 +78.21 x_2-296.847 &= 0 
\end{split}
\label{nuevomod3afinish}
\end{equation}

Al simular ambos modelos en matlab con la condición inicial dada obtuvimos la siguiente gráfica:
\begin{figure}[H]
    \centering
    \begin{subfigure}[H]{0.45\linewidth}
        \includegraphics[width=\linewidth]{plots3/plots3a/3ax_1.png}
        \caption{\centering Soluciones para $x_1$}
        \label{}
    \end{subfigure}
    \hspace{0.5cm}
    \begin{subfigure}[H]{0.45\linewidth}
        \includegraphics[width=\linewidth]{plots3/plots3a/3ax_2.png}
        \caption{\centering Soluciones para $x_2$}
        \label{fig:config 2}
    \end{subfigure}
    \caption{Soluciones para el modelo del punto 3a}
\end{figure}

Las gráficas muestran las soluciones numéricas para la variable $x_1$ y $x_2$ del modelo linealizado y no linealizado, estas gráficas están definidas hasta t=0.6 debido a que hubo problemas con la compilación después de ese valor ya que el resultado de una de las derivadas del modelo tendía a infinito. Sin embargo en el rango que se pudo graficar podemos observar que las curvas de los resultados son bastante diferentes, por lo que podemos llegar a la conclusión de que, si bien pudimos volver lineal el modelo, los resultados de las variables se ven afectados gravemente por esta linealización.



\item $y(1)=2,\bar{y}=10$
\begin{equation}
     \sqrt[]{y}\frac{dy}{dx}+y^{1.5}-1 = 0
\end{equation}
% SOLUCIÓN
\textbf{Solución:}
El modelo es no lineal debido al producto la raíz de la variable dependiente con la derivada de la variable dependiente y por el término $y^{1.5}$.
Para ver mejor el sistema cambiamos la notación así:

\begin{equation}
     y^{\frac{1}{2}}\dot{y}+y^{\frac{3}{2}}-1 = 0
     \label{3bnotacion}
\end{equation}

Si dividimos (\ref{3bnotacion}) entre $y^{\frac{1}{2}}$ (poniendo como condición que $y\neq0$) obtenemos:

\begin{equation}
     \dot{y}+y-y^{-\frac{1}{2}} = 0
     \label{3bdivid}
\end{equation}

En (\ref{3bdivid}) vemos que el único término no líneal es $-y^{-\frac{1}{2}}$ así que procedemos a linealizarlo haciendo uso de las series de Taylor asumiendo los puntos de operación indicados.

\begin{equation}
    y^{-\frac{1}{2}}\simeq \frac{10^{-\frac{1}{2}}}{0!}-\frac{10^{-\frac{3}{2}}}{{2(1!)}}(y-10)+3\frac{10^{-\frac{5}{2}}}{{4(2!)}}(y-10)^2+...
    \label{3bsumat}
\end{equation}

Ya que buscamos linealizar el modelo tomaremos solamente los dos primeros términos de la sumatoria (\ref{3bsumat}), es decir:

\begin{equation}
    y^{-\frac{1}{2}}\simeq \frac{10^{-\frac{1}{2}}}{0!}-\frac{10^{-\frac{3}{2}}}{{2(1!)}}(y-10)
    \label{3bsumatsimplif}
\end{equation}

Al simplificar (\ref{3bsumatsimplif}) obtenemos que:

\begin{equation}
    y^{-\frac{1}{2}}\simeq 0.4743-0.0158y
    \label{3by-1/2}
\end{equation}

Reemplazando (\ref{3by-1/2}) en (\ref{3bdivid}) obtenemos el modelo equivalente linealizado que sería:


\begin{equation}
    \dot{y}+1.0158y-0.4743=0
\end{equation}

Al simular ambos modelos en matlab con la condición inicial dada obtuvimos la siguiente gráfica:
\begin{figure}[H]
    \centering
    \includegraphics[width=0.6\linewidth]{plots3/plots3b/3bmatlab.png}
    \caption{Soluciones para el modelo del punto 3b.}
    \label{3b}
\end{figure}

La gráfica muestra las soluciones numéricas del modelo linealizado y no linealizado, esta gráfica está definida desde x=1 debido a la condición inicial definida para el ejercicio, vemos que el tamaño las curvas de los resultados son diferentes entre ellas, sin embargo, tienen ``forma similar". También observamos que ambas cumplen la condición que planteamos de $y\neq 0$. 

\item Para este caso solamente linealizarlo. Punto de operación: $ \bar{x} = \bar{y} = \bar{z} = 1$
\begin{equation}
    f(x,y,z)=\cos(x+y)^z+e^{(x+y+z)^2}-\cosh{\sqrt{x+y+z}-\left[\frac{x+y}{z}\right]^x}
    \label{p3cec}
\end{equation}

% SOLUCIÓN
\textbf{Solución}

El modelo es no lineal debido a todos sus términos.

Para linealizar el modelo, tomamos cada función no lineal dentro de el mismo, y procedemos a linealizarlas por separado haciendo uso de las series de Taylor:

\begin{multline}
    \notag
    \cos((x+y)^z)\simeq \cos((x+y)^z)-z(x+y)^{z-1} \sin((x+y)^z)(x-\bar{x}) ...
    \\...-z(x+y)^{z-1} \sin((x+y)^z)(y-\bar{y})
    -(x+y)^z \ln(x+y) \sin((x+y)^z)(z-\bar{z})
\end{multline}
\begin{multline}
    \notag
    e^{(x+y+z)^2}\simeq e^{(x+y+z)^2}+2(x+y+z)e^{(x+y+z)^2}(x-\bar{x})+2(x+y+z)e^{(x+y+z)^2}(y-\bar{y})...
    \\...+2(x+y+z)e^{(x+y+z)^2}(z-\bar{z})
\end{multline}
\begin{multline}
    \notag
    \cosh(\sqrt{x+y+z})\simeq \cosh(\sqrt{x+y+z})+\frac{\sinh(\sqrt{x+y+z})}{2\sqrt{x+y+z}}(x-\bar{x})...
    \\...+\frac{\sinh(\sqrt{x+y+z})}{2\sqrt{x+y+z}}(y-\bar{y})+\frac{\sinh(\sqrt{x+y+z})}{2\sqrt{x+y+z}}(z-\bar{z})
\end{multline}
\begin{multline}
    \notag
    \left[\frac{x+y}{z}\right]^x\simeq \left[\frac{x+y}{z}\right]^x +\left[\frac{x+y}{z}\right]^x \left(\ln\left[\frac{x+y}{z}\right]^x+\frac{x}{x+y}\right)(x-\bar{x})+\frac{x\left[\frac{x+y}{z}\right]^x}{x+y}(y-\bar{y})...
    \\ ...-\frac{x\left[\frac{x+y}{z}\right]^x}{z}(z-\bar{z})
\end{multline}

 Al reemplazar en los puntos de operación $ \bar{x} = \bar{y} = \bar{z} = 1$, obtenemos: 
\begin{equation}
    \cos((x+y)^z)\simeq 2.663001-0.909297x-0.909297y-1.260554z
    \label{p3cec1}
\end{equation}
\begin{equation}
     e^{(x+y+z)^2}\simeq 48618.50357x+48618.50357y+48618.50357z-137752.4268
     \label{p3cec2}
\end{equation}
\begin{equation}
     \cosh(\sqrt{x+y+z})\simeq 0.790293x+0.790293y+0.790293z+0.543698
     \label{p3cec3}
\end{equation}
\begin{equation}
     \left[\frac{x+y}{z}\right]^x\simeq 2.386294x+y-2z+0.613706
     \label{p3cec4}
\end{equation}
Reemplazamos (\ref{p3cec1}),(\ref{p3cec2}),(\ref{p3cec3}) y (\ref{p3cec4}) en (\ref{p3cec})

\begin{equation}
    f(x,y,z)\simeq 48614.417686x+48615.80398y+48618.452723z-137750.921203
\end{equation}

\end{enumerate}
\item Apliquen el método de la impedancia (si es posible) para modelar el siguiente sistema. En caso de no poderse, explicar claramente el por qué no.

Si es viable, deben detallar paso a paso cómo se hizo. No valen respuestas “sorpresa”. El resorte es lineal, tiene una longitud inicial $l_0$ y una constante $k$. La polea tiene un radio $r$, una masa $M$ y un momento de inercia $I = \frac{1}{2}Mr^2$
\begin{figure}[H]
    \centering
    \includegraphics[width=0.5\linewidth]{sistema punto 4.png}
    \caption{Sistema masa-polea-resorte a modelar.}
    \label{fig:sistema punto 4}
\end{figure}

Inicialmente definimos el torque (T) de la polea
    \begin{equation}
    T=\ddot{\theta}I \left[Nm\right]
    \end{equation}
Tenemos que el torque en este caso es igual a la fuerza por el radio de la polea,así que lo escribiremos de la siguiente forma:
    \begin{equation}
    \ddot{\theta}\frac{1}{2}Mr^2=Fr
    \label{torque}
    \end{equation}
Ahora, sabemos que podemos obtener la longitud de un arco usando el ángulo y el rádio de este, así: $\theta r=x$.
En esta expresión si despejamos $\theta$ y derivamos dos veces con respecto al tiempo obtenemos que:
    \begin{equation}
    \ddot{\theta}=\frac{\ddot{x}}{r}
    \label{aceleracionang}
    \end{equation}
si reemplazamos (\ref{aceleracionang}) en (\ref{torque}) obtenemos que:
\begin{equation}
\notag
   \frac{\ddot{x}}{r}\frac{1}{2}Mr^2=Fr
    \end{equation}
y finalmente al despejar la fuerza obtenemos que: 
\begin{equation}
   F=\ddot{x}\frac{1}{2}M  \left[N\right]
    \end{equation}
    
Del método de impedancias mecánicas sabemos que una fuerza de este tipo corresponde a la fuerza sobre una masa, por lo que podemos hacer un modelo equivalente del sistema usando una masa equivalente de $\frac{1}{2}M+m$ [kg].

\begin{figure}[H]
\centering
\begin{tikzpicture}
% Supporting structure
\fill [pattern = north west lines] (-1.5,0) rectangle ++(3,.5);
\draw (-1.5,0) -- ++(3,0);
\draw
[
    decoration={
        coil,
        segment length = 2mm,
        amplitude = 5mm,
        aspect = 0.3,
        post length = 4mm,
        pre length = 4mm},
    decorate] (0,0) -- ++(0,-4)
   node[midway,right=0.5cm,black]{$k$}; 
% Mass
\node[draw,
    fill=yellow!60,
    minimum width=1cm,
    minimum height=1cm,
    anchor=north,
    label=east:$\frac{1}{2}M+m$] at (0,-4) {};
    %Elongacion inicial
    \draw[very thick,
    red,
    |-|] (-2,0) -- ++(0,-4)
    node[midway,left]{\small $l_0$};
    %Eje de referencia
    \draw [very thick,
    blue,
    -latex
] (-0.8,-4.5) -- ++(-1,0) ++(0.25,0) -- ++ (0,-1)
    node[midway,left]{\small $x$};
 \end{tikzpicture}
 \caption{Sistema masa-resorte equivalente.}
\end{figure}
Aplicando el método de impedancias en el dominio del tiempo tenemos que el sistema se modelaría de la siguiente manera:
\begin{equation}
\mathbf{
    \left(\frac{1}{2}M+m\right)\ddot{x}+kx=0}
\end{equation}


\end{enumerate}
\end{document}
