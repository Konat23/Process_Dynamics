\documentclass[letterpaper, 12pt]{article}
\usepackage[utf8]{inputenc}
\usepackage{amsmath,amssymb,amsfonts}
\usepackage[spanish]{babel}
\spanishdecimal{.}
\usepackage{algorithmic}
\usepackage{graphicx}
\usepackage{subcaption}
\usepackage{textcomp}
\usepackage{xcolor}
\usepackage{enumerate}
\usepackage{tikz} % Dibujar resortes
\usetikzlibrary{decorations.pathmorphing,patterns}%también para los resortes xd
\usepackage[left=1in, right=1in, top=1in, bottom=1in]{geometry}

\usepackage{float} % Ubicar figuras HERE
\usepackage{soul}  % Resaltador

\pagecolor{white}
\color{black}% set the default colour to white

\title{\protect\vspace{-0.5cm}\protect\includegraphics[width =6cm]{Logo UIS.png} \\
Primer Parcial - Dinámica de Procesos}
\author{Sebastián García Angarita - 2191472\and
        Juan Sebastián Guerrero Peña - 2190435\and
        Rafael Santiago Suarez Gil - 2190433\and
        Luis Fernando Romero Rojas - 2191663 \\ \\
        \textit{Escuela de Ingeniería Eléctrica, Electrónica y Telecomunicaciones} \\
        \textit{Universidad Industrial de Santander}\\
        Bucaramanga, Colombia \\}
\date{}

\begin{document}
\maketitle
\begin{center}\rule{0.9\linewidth}{0.5pt}\end{center}
\parskip = 12pt

\hl{TODO: Hacer un listado de las unidades de las variables que se usan. Creo que solo podremos usar unidades en aquellas en la que se nos especifica las variables en las que esta, como el punto 4}

\begin{enumerate}
\item Desarrollen numéricamente los siguientes modelos matemáticos, grafique los resultados y analícelos:
\begin{enumerate}
\item Utilicen los siguientes valores:
$L = 1.21$; $g = 10$; $a_1 = 2.5$; $a_2 = 5.0$. Asuma condiciones iniciales diferentes de cero para $t(0)$. El modelo es dimensionalmente consistente:
\begin{equation}
\begin{split}
    L^2\ddot{\varphi} - \ddot{\gamma}\sin \varphi + \ddot{\mu}\cos \varphi + \cos \varphi &= 0 \\
    \frac{a_1+a_2}{a_2L}\ddot{\mu} - \ddot{\varphi}\cos \varphi - \dot{\varphi}^2\sin \varphi &= - \frac{a_1+a_2}{a_2L}g \\
    \frac{a_1+a_2}{a_2L}\ddot{\gamma} - \ddot{\varphi}\sin \varphi &= \dot{\varphi}^2\cos \varphi
\end{split}
\end{equation}

% SOLUCIÓN






\item $g = 10$; $m_1 = 10$; $m_2 = 30$; $k = 30$; $r_1 = 2.0$. El modelo es dimensionalmente consistente.
\begin{equation}
\begin{split}
    \left(m_1{r_1}^2 + m_2{r_2}^2\right)\ddot{\theta} + 2m_2r_2\dot{r_2}\dot{\theta} + \left(m_1r_1 + m_2r_2\right)g\sin \theta &= 0 \\
    m_2\ddot{r_2} - m_2r_2{\dot{\theta}}^2 - m_2g\cos \theta + k\left(r_2 - 1.1\right) &= 0
\end{split}
\end{equation}









\item Para el siguiente modelo asuma que $J_1 < J_2 < J_3$. Proponga valores para estos momentos de inercia cumpliendo la desigualdad y analice. Realice al menos cinco simulaciones con diferentes $J$'s manteniendo las mismas condiciones iniciales (que deben ser diferentes a cero, para el tiempo $t_0 = 0$). $\omega_i \equiv \omega_i(t)$,
\begin{equation}
\begin{split}
    J_1\dot{\omega_1} + \left( J_2 - J_3\right)\omega_2\omega_3 &= 0 \\
    J_2\dot{\omega_2} - \left( J_1 - J_3\right)\omega_1\omega_3 &= 0 \\
    J_3\dot{\omega_3} + \left( J_1 - J_2\right)\omega_2\omega_1 &= 0
\end{split}
\end{equation}

% SOLUCIÓN
\textbf{Solución}

Estas ecuaciones describen la rotación de un objeto que tiene tres momentos de inercia distintos en sus tres ejes principales.
Cuando el objeto rota únicamente sobre uno de los ejes, se dice que el objeto está en equilibrio porque la velocidad angular no cambia y permanece igual que la velocidad inicial como se puede ver en la figura \ref{fig:sistema en punto de equilibrio}.

\begin{figure}[h!]
    \centering
    \hspace{1cm}
    \begin{subfigure}[h]{0.28\linewidth}
        \includegraphics[width=\linewidth]{plots 1c/w0=[1,0,0].png}
        \caption{\centering Rotación en el primer eje}
    \end{subfigure}
    \begin{subfigure}[h]{0.28\linewidth}
        \includegraphics[width=\linewidth]{plots 1c/w0=[0,1,0].png}
        \caption{\centering Rotación en el segundo eje}
    \end{subfigure}
    \begin{subfigure}[h]{0.28\linewidth}
        \includegraphics[width=\linewidth]{plots 1c/w0=[0,0,1].png}
        \caption{\centering Rotación en el tercer eje}
    \end{subfigure}
    \caption{Velocidades angulares cuando el sistema está en un punto de equilibrio}
    \label{fig:sistema en punto de equilibrio}
\end{figure}

Estos resultados son intuitivos, sin presencia de fuerzas externas el cuerpo permanece rotando sobre el mismo eje a la misma velocidad.

Ahora un caso más interesante, es cuando el cuerpo tiene una velocidad inicial en el segundo eje, pero tiene perturbaciones muy pequeñas en los otros ejes. Este es el conocido como teorema del eje intermedio.

Para fines del ejercicio, se analiza el comportamiento al cambiar el momento de inercia de los ejes. Por tanto se establecen las siguientes velocidades iniciales de cada eje:
$\omega_1(0) = 0.001$; $\omega_2(0) = 1$; $\omega_3(0) = 0.001$.

Para el primer caso se establecieron los momentos angulares así:
$$J_1 = 2;\hspace{0.25cm}$$
Js = [2 4 7;
      1 4 7;
      2 5 7;
      2 3 7;
      2 4 9];

\begin{figure}[h!]
    \centering
    \hspace{1cm}
    \begin{subfigure}[h]{0.3\linewidth}
        \includegraphics[width=\linewidth]{plots 1c/w_i vs t - v1.png}
        \caption{\centering Rotación en el primer eje}
    \end{subfigure}
    \begin{subfigure}[h]{0.3\linewidth}
        \includegraphics[width=\linewidth]{plots 1c/w_i vs t - v2.png}
        \caption{\centering Rotación en el segundo eje}
    \end{subfigure}
    \begin{subfigure}[h]{0.3\linewidth}
        \includegraphics[width=\linewidth]{plots 1c/w_i vs t - v3.png}
        \caption{\centering Rotación en el tercer eje}
    \end{subfigure}
    \begin{subfigure}[h]{0.3\linewidth}
        \includegraphics[width=\linewidth]{plots 1c/w_i vs t - v4.png}
        \caption{\centering Rotación en el tercer eje}
    \end{subfigure}
    \begin{subfigure}[h]{0.3\linewidth}
        \includegraphics[width=\linewidth]{plots 1c/w_i vs t - v5.png}
        \caption{\centering Rotación en el tercer eje}
    \end{subfigure}
    \caption{Velocidades angulares cuando el sistema está en un punto de equilibrio}
    \label{fig:sistema en punto de equilibrio}
\end{figure}





\end{enumerate}
\item Representen es ESPACIO DE ESTADOS (en caso de que se pueda) los siguientes modelos:
\begin{enumerate}
\item Se medirán todas las salidas con sendos sensores junto con las fuentes.
\begin{equation}
\begin{split}
    D\ddot{\varphi} - \ddot{\gamma}\sin \varphi + \ddot{\mu}\cos \varphi + \cos \varphi &= 0 \\
    A\ddot{\mu} - \ddot{\varphi}\cos \varphi - {\dot{\varphi}}^2\sin \varphi &= C \\
    B\ddot{\gamma} - \ddot{\varphi}\sin \varphi &= {\dot{\varphi}}^2\cos \varphi
\end{split}
\end{equation}

% SOLUCIÓN








\item Aquí se medirán simultáneamente todas las salidas con sendos sensores, e incluso las mismas entradas (pero con sensores diferentes y únicos (¡dedicación exclusiva para cada entrada!)).

\begin{equation}
\begin{split}
    55\dddot{x}_1-\ddot{x}_2-7\dot{x}_3-5x_2-7\sin(0.1t)=0 \\
    {\dddot{x}}_2-\ {\ddot{x}}_1\ +18{\dot{x}}_2\ +\ 4x_3\ +\ 7\cosh(0.5t)=0 \\
    \pi\dddot{x}_3\ -\ 6\ddot{x}_3\ -\ 7\dot{x}_1\ -\ 8x_2\ -\ 3\tan(0.7t) =0 
\end{split}
\end{equation}

\textbf{Solución} \\
Identificamos 3 variables de estado las cuales estan todas como tercera derivadas por lo tanto necesitamos usar cambios de variable para obtener las 6 variables de fase que los complementan:
\begin{equation}
\begin{split}
    y_1 &= \dot{x}_1 \\
    y_2 &= \dot{x}_2 \\
    y_3 &= \dot{x}_3 \\
    z_1 &= \dot{y}_1 \\
    z_2 &= \dot{y}_2 \\
    z_3 &= \dot{y}_3 \label{2.b.CambioV}
\end{split} 
\end{equation}
Podemos entonces reescribir las ecuaciones como:
\begin{align*}
    55\dot{z}_1-z_2-7y_3-5x_2 &= 7\sin(0.1t) \\
    {\dot{z}}_2-\ z_1\ +18y_2\ +\ 4x_3\ &=\ -7 \cosh(0.5t) \\
    \pi\dot{z}_3\ -\ 6\ z_3\ -\ 7\ y_1\ -\ 8x_2\ &=\ 3\tan(0.7t) \   
\end{align*}
Si despejamos de cada una de las ecuaciones las derivadas de las variables de estado quedaría:
 \begin{equation}
 \begin{split}
    \dot{z}_1&=\frac{1}{55}z_2+\frac{7}{55}y_3+\frac{5}{55}x_2+\frac{7}{55}\sin(0.1t) \\
    {\dot{z}}_2&=\ z_1\ -18y_2\ -\ 4x_3\ \ -7\cosh(0.5t)  \\
    \dot{z}_3\ &=\ \frac{6}{\pi}\ z_3\ +\ \frac{7}{\pi}\ y_1\ +\ \frac{8}{\pi}x_2\ +\ \frac{8}{\pi}\tan(0.7t)
    \label{1.b.mainV}
\end{split}
\end{equation}

Si se usan las ecuaciones (\ref{2.b.CambioV}) y (\ref{1.b.mainV}) podemos formar la matriz: \
\begin{multline*}
\begin{bmatrix}
\dot{z}_1\\ 
\dot{z}_2\\ 
\dot{z}_3\\ 
\dot{y}_1\\ 
\dot{y}_2\\ 
\dot{y}_3\\ 
\dot{x}_1\\ 
\dot{x}_2\\ 
\dot{x}_3
\end{bmatrix}
=
\begin{bmatrix}
0 &1/55  &0  &0  &0  &7/55  &0  &5/55  &0 \\ 
1 &0  &0  &0  &-18  &0  &0  &0  &-4 \\ 
0 &0  &6/\pi &7/\pi   &0  &0  &0  &8/\pi   &0 \\ 
1 &0  &0  &0  &0  &0  &0  &0  &0 \\ 
0 &1  &0  &0  &0  &0  &0  &0  &0 \\ 
0 &0  &1  &0  &0  &0  &0  &0  &0 \\ 
0 &0  &0  &1  &0  &0  &0  &0  &0 \\ 
0 &0  &0  &0  &1  &0  &0  &0  &0 \\ 
0 &0  &0  &0  &0  &1  &0  &0  &0 
\end{bmatrix}
\begin{bmatrix}
z_1\\ 
z_2\\ 
z_3\\ 
y_1\\ 
y_2\\ 
y_3\\ 
x_1\\ 
x_2\\ 
x_3
\end{bmatrix}
\cdots \\
\cdots +
\begin{bmatrix}
7/55 &0  &0   \\ 
0 &-7  &0   \\ 
0 &0  &3/\pi \\
0  &0  &0 \\
0  &0  &0 \\
0  &0  &0 \\
0  &0  &0 \\
0  &0  &0 \\
0  &0  &0 \\
\end{bmatrix}
\begin{bmatrix}
\sin(0.1t)\\ 
\cosh(0.5t)\\ 
\tan(0.7t)
\end{bmatrix}
\end{multline*}
La cual tiene la forma de la \textbf{ecuación de estado}: $\dot{x}=Ax+Bu$. \\
Como se requiere medir todas las salidas y todas las entradas por independiente nuestra ecuación de salida tiene que ser un vector de $12\cdot 1$ donde 9 espacios son ocupados por las variables de estado y 3 por las entradas, tendríamos entonces:


\begin{multline*}
\begin{bmatrix}
\ y_1\\ 
\ y_2\\ 
\ y_3\\ 
\ y_4\\ 
\ y_5\\ 
\ y_6
\end{bmatrix}
=
\begin{bmatrix}
0 &0  &0  &0  &0  &0  &1  &0  &0 \\ 
0 &0  &0  &0  &0  &0  &0  &1  &0 \\ 
0 &0  &0  &0  &0  &0  &0  &0  &1 \\ 
0 &0  &0  &0  &0  &0  &0  &0  &0 \\ 
0 &0  &0  &0  &0  &0  &0  &0  &0 \\ 
0 &0  &0  &0  &0  &0  &0  &0  &0 
\end{bmatrix}
\begin{bmatrix}
z_1\\ 
z_2\\ 
z_3\\ 
y_1\\ 
y_2\\ 
y_3\\ 
x_1\\ 
x_2\\ 
x_3
\end{bmatrix}
\cdots \\
\cdots +
\begin{bmatrix}
0  &0  &0 \\
0  &0  &0 \\
0  &0  &0 \\
1  &0  &0 \\
0  &1  &0 \\
0  &0  &1 
\end{bmatrix}
\begin{bmatrix}
\sin(0.1t)\\ 
\cosh(0.5t)\\ 
\tan(0.7t)
\end{bmatrix}
\end{multline*}









\item Se medirán simultáneamente todas las salidas con sendos sensores (incluyendo las entradas).
\begin{equation}
\begin{split}
    12{\dddot{x}}_1\ -\ {\ddot{x}}_2\ +\ t{\dot{x}}_5\ -\ 5{\dot{x}}_4\ -\ \sin(0.1t) &=0 \\
    \dddot{x}_2\ -\ t\ddot{x}_6\ +\ 18\dot{x}_2\ + 4x_3 + x_1 + t \cosh(0.5) &= 0 \\
    6{\ddot{x}}_5-\ t\ -1{\dot{x}}_1+\ 8{\ddot{x}}_2-\ 3\tan(t + 1)x_6 &= 0\\
    5{\dddot{x}}_4\ -\ {\ddot{x}}_2\ -\ 7{\ddot{x}}_3\ -\cos(t)x_1\ +\ 0.1 &= 0 \\ 
    12{\dddot{\ x}}_5-\ {\ddot{x}}_1+\ t\ 3{\dot{x}}_2+\ 4x_4+\ x_3\cosh(t) &=0 \\
    -t{\dddot{x}}_6\ -\ e\ -5t{\ddot{x}}_3-\ 3{\dot{x}}_1-\ 8x_5+\ t\tan(t + 2) &= 0
\end{split}
\end{equation}

% SOLUCIÓN


\end{enumerate}
\item Linealice los siguientes modelos (en caso de ser no-lineales) y resuélvanlos numéricamente. Definan las condiciones de solución del sistema. Grafiquen las soluciones del sistema no-lineal y el
correspondiente linealizado, y analicen. Incluya todos los pasos de la linealización. 
\begin{enumerate}
\item Linealizarlo alrededor del punto de operación: $\bar{x}_1=\bar{x}_2=5$.
\begin{equation}
\begin{split}
    \dddot{x}_1-\ddot{x}_1+18\ddot{x}_2+4x_1+x_2+\cosh(x_1) &= 0 \\ \dddot{x}_2-\ddot{x}_2+18\ddot{x}_1+4x_2+x_1+\sinh(x_2) &= 0
\end{split}
\label{mod1}
\end{equation}

% SOLUCIÓN
\textbf{Solución:}
El modelo es no lineal debido al coseno y al seno hiperbólico que afecta a las variables dependientes, para linealizar al rededor de $\bar{x}_1=\bar{x}_2=5$ hacemos $x_1=\delta x_1+5$ y $x_2=\delta x_2+5$ donde $\delta x_1$ y $\delta x_2$ son pequeñas excursiones al rededor de $\bar{x}_1=\bar{x}_2=5$

\begin{equation}
    \notag
    \cosh(x_1)\simeq \frac{\cosh(5)}{0!}+\frac{\senh(5)}{1!}\left(x_1-5\right)+\frac{\cosh(5)}{2!}\left(x_1-5\right)^2+\frac{\senh(5)}{3!}\left(x_1-5\right)^3+ \cdots
\end{equation}
\begin{equation}
    \notag
    \senh(x_2)\simeq \frac{\senh(5)}{0!}+\frac{\cosh(5)}{1!}\left(x_2-5\right)+\frac{\senh(5)}{2!}\left(x_2-5\right)^2+\frac{\cosh(5)}{3!}\left(x_2-5\right)^3+ \cdots
\end{equation}

Ya que buscamos linealizar el modelo tomaremos solamente los dos primeros términos de la sumatoria, es decir:
\begin{equation}
    \notag
    \cosh(x_1)\simeq \frac{\cosh(5)}{0!}+\frac{\senh(5)}{1!}\left(x_1-5\right)
\end{equation}
\begin{equation}
    \notag
    \senh(x_2)\simeq \frac{\senh(5)}{0!}+\frac{\cosh(5)}{1!}\left(x_2-5\right)
\end{equation}

Al resolver obtenemos que

\begin{equation}
    \cosh(x_1)\simeq 74.203x_1-296.805
    \label{cosh}
\end{equation}

\begin{equation}
    \senh(x_2)\simeq 74.21x_2-296.847
    \label{senh}
\end{equation}

El nuevo modelo obtenido al reemplazar (\ref{cosh}) y (\ref{senh}) en (\ref{mod1}) es:

\begin{equation}
\begin{split}
    \dddot{x}_1-\ddot{x}_1+18\ddot{x}_2+4x_1+x_2+74.203x_1-296.805 &= 0 \\ \dddot{x}_2-\ddot{x}_2+18\ddot{x}_1+4x_2+x_1+74.21x_2-296.847 &= 0
\end{split}
\end{equation}

\item $y(1)=2,\bar{y}=10$
\begin{equation}
     \sqrt[]{y}\frac{dy}{dx}+y^{1.5}-1 = 0
\end{equation}
% SOLUCIÓN
\textbf{Solución:}
El modelo es no lineal debido al producto la raíz de la variable dependiente con la derivada de la variable dependiente y por el término $y^1.5$.
Para ver mejor el sistema cambiamos la notación así:

\begin{equation}
     y^{\frac{1}{2}}\dot{y}+y^{\frac{3}{2}}-1 = 0
     \label{3bnotacion}
\end{equation}

Si dividimos (\ref{3bnotacion}) entre $y^{\frac{1}{2}}$ obtenemos:

\begin{equation}
     \dot{y}+y-y^{-\frac{1}{2}} = 0
     \label{3bdivid}
\end{equation}

En (\ref{3bdivid}) vemos que el único término no líneal es $-y^{-\frac{1}{2}}$ así que procedemos a linealizarlo haciendo uso de las series de Taylor asumiendo los puntos de operación indicados.

\begin{equation}
    y^{-\frac{1}{2}}\simeq \frac{10^{-\frac{1}{2}}}{0!}-\frac{10^{-\frac{3}{2}}}{{2(1!)}}(y-10)+3\frac{10^{-\frac{5}{2}}}{{4(2!)}}(y-10)^2+...
    \label{3bsumat}
\end{equation}

Ya que buscamos linealizar el modelo tomaremos solamente los dos primeros términos de la sumatoria (\ref{3bsumat}), es decir:

\begin{equation}
    y^{-\frac{1}{2}}\simeq \frac{10^{-\frac{1}{2}}}{0!}-\frac{10^{-\frac{3}{2}}}{{2(1!)}}(y-10)
    \label{3bsumatsimplif}
\end{equation}

Al simplificar (\ref{3bsumatsimplif}) obtenemos que:

\begin{equation}
    y^{-\frac{1}{2}}\simeq 0.4743-0.0158y
    \label{3by-1/2}
\end{equation}

Reemplazando (\ref{3by-1/2}) en (\ref{3bdivid}) obtenemos el modelo equivalente linealizado que sería:


\begin{equation}
    \dot{y}+1.0158y-0.4743=0
\end{equation}

\item Para este caso solamente linealizarlo. Punto de operación: $ \bar{x} = \bar{y} = \bar{z} = 1$
\begin{equation}
    f(x,y,z)=\cos(x+y)^z+e^{(x+y+z)^2}-\cosh{\sqrt{x+y+z}-\left[\frac{x+y}{z}\right]^x}
\end{equation}

% SOLUCIÓN





\end{enumerate}
\item Apliquen el método de la impedancia (si es posible) para modelar el siguiente sistema. En caso de no poderse, explicar claramente el por qué no.

Si es viable, deben detallar paso a paso cómo se hizo. No valen respuestas “sorpresa”. El resorte es lineal, tiene una longitud inicial $l_0$ y una constante $k$. La polea tiene un radio $r$, una masa $M$ y un momento de inercia $I = \frac{1}{2}Mr^2$
\begin{figure}[H]
    \centering
    \includegraphics[width=0.5\linewidth]{sistema punto 4.png}
    \caption{Sistema masa-polea-resorte a modelar.}
    \label{fig:sistema punto 4}
\end{figure}

Inicialmente definimos el torque (T) de la polea
    \begin{equation}
    T=\ddot{\theta}I \left[Nm\right]
    \end{equation}
Tenemos que el torque en este caso es igual a la fuerza por el radio de la polea,así que lo escribiremos de la siguiente forma:
    \begin{equation}
    \ddot{\theta}\frac{1}{2}Mr^2=Fr
    \label{torque}
    \end{equation}
Ahora, sabemos que podemos obtener la longitud de un arco usando el ángulo y el rádio de este, así: $\theta r=x$.
En esta expresión si despejamos $\theta$ y derivamos dos veces con respecto al tiempo obtenemos que:
    \begin{equation}
    \ddot{\theta}=\frac{\ddot{x}}{r}
    \label{aceleracionang}
    \end{equation}
si reemplazamos (\ref{aceleracionang}) en (\ref{torque}) obtenemos que:
\begin{equation}
\notag
   \frac{\ddot{x}}{r}\frac{1}{2}Mr^2=Fr
    \end{equation}
y finalmente al despejar la fuerza obtenemos que: 
\begin{equation}
   F=\ddot{x}\frac{1}{2}M  \left[N\right]
    \end{equation}
    
Del método de impedancias mecánicas sabemos que una fuerza de este tipo corresponde a la fuerza sobre una masa, por lo que podemos hacer un modelo equivalente del sistema usando una masa equivalente de $\frac{1}{2}M+m$ [kg].

\begin{figure}[H]
\centering
\begin{tikzpicture}
% Supporting structure
\fill [pattern = north west lines] (-1.5,0) rectangle ++(3,.5);
\draw (-1.5,0) -- ++(3,0);
\draw
[
    decoration={
        coil,
        segment length = 2mm,
        amplitude = 5mm,
        aspect = 0.3,
        post length = 4mm,
        pre length = 4mm},
    decorate] (0,0) -- ++(0,-4)
   node[midway,right=0.5cm,black]{$k$}; 
% Mass
\node[draw,
    fill=yellow!60,
    minimum width=1cm,
    minimum height=1cm,
    anchor=north,
    label=east:$\frac{1}{2}M+m$] at (0,-4) {};
    %Elongacion inicial
    \draw[very thick,
    red,
    |-|] (-2,0) -- ++(0,-4)
    node[midway,left]{\small $l_0$};
    %Eje de referencia
    \draw [very thick,
    blue,
    -latex
] (-0.8,-4.5) -- ++(-1,0) ++(0.25,0) -- ++ (0,-1)
    node[midway,left]{\small $x$};
 \end{tikzpicture}
 \caption{Sistema masa-resorte equivalente.}
\end{figure}
Aplicando el método de impedancias en el dominio del tiempo tenemos que el sistema se modelaría de la siguiente manera:
\begin{equation}
\mathbf{
    \left(\frac{1}{2}M+m\right)\ddot{x}+kx=0}
\end{equation}


\end{enumerate}
\end{document}
